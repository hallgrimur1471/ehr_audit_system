\documentclass[11pt]{article}
\usepackage[T1]{fontenc}
\usepackage[utf8]{inputenc}
\usepackage[english]{babel}
\usepackage[normalem]{ulem}
\usepackage{amsmath}
\usepackage{xcolor}
\usepackage{listings}
\usepackage{listingsutf8}
\usepackage{lstautogobble}
\usepackage{titling}
\usepackage{fancyhdr}
\usepackage{graphicx}
\usepackage{underscore}
\usepackage{hyperref}
\graphicspath{{images/}}

\newcommand{\headerhaegri}{}
\newcommand{\headermidja}{University of Southern California}
\newcommand{\headervinstri}{}
\newcommand{\headerhaed}{14pt}
\newcommand{\headertykkt}{0.4pt}

% Title
\newcommand{\nafnanamskeidi}{CSCI531 Applied Cryptogrophy\\Final Project}
\newcommand{\nafnaverkefni}{EHR Audit System}
\newcommand{\nafnanemanda}{Student: Hallgrimur David Egilsson\\USC ID: 6059-2639-79\vspace{4mm}}
\newcommand{\nafnakennara}{Professor: Dr. Tatyana Ryutov}
\newcommand{\dagsetning}{\today}
\setlength{\droptitle}{-20mm}
\newcommand{\haednemanda}{-2mm}
\newcommand{\haeddagsetningar}{0mm}
\newcommand{\haedtexta}{0mm}

% Page 1 footer
\newcommand{\footervinstri}{}
\newcommand{\footermidja}{}
\newcommand{\footerhaegri}{}
\newcommand{\footertykkt}{0.4pt}

% Page 2,3,4,... header
\lhead{CSCI 531}
\chead{}
\rhead{\nafnaverkefni}
\renewcommand{\headrulewidth}{0.4pt}

% Page 2,3,4,... footer
\lfoot{}
\cfoot{\thepage}
\rfoot{}
\renewcommand{\footrulewidth}{0.4pt}
\usepackage[top = 1.8in
		   ,bottom=1.715in
		   ,left=1.76in
		   ,right=1.76in
		   ]{geometry}


\let\oldfrac\frac
\let\frac\dfrac

\pagestyle{fancy}
\fancypagestyle{bls1}
{
	\fancyhf{}
	\lhead{\headervinstri}
	\chead{\headermidja}
	\rhead{\headerhaegri}
	\headheight = \headerhaed
	\lfoot{\footervinstri}
	\cfoot{\footermidja}
	\rfoot{\footerhaegri}
	\renewcommand{\headrulewidth}{\headertykkt}
	\renewcommand{\footrulewidth}{\footertykkt}
}

\title{
\huge\nafnanamskeidi\\
\vspace{8mm}
\nafnaverkefni\vspace{\haednemanda}}
\author{\nafnanemanda\\
        \nafnakennara\vspace{\haeddagsetningar}}
\date{\dagsetning\vspace{\haedtexta}}

\setlength{\parindent}{4em}
\setlength{\parskip}{4mm}

\renewcommand{\lstlistingname}{}
\lstset{frame=lines}
\lstset{caption={}}
\lstset{label={lst:code_direct}}
\lstset{basicstyle=\footnotesize}
\lstset{autogobble=true}
\lstset{literate = {-}{-}1}
\lstset{inputencoding=utf8/latin9}

\begin{document}
	\maketitle
	\thispagestyle{bls1}
	
	\begin{flushleft}
		
		\vspace{24mm}
		I have read the Guide to Avoiding Plagiarism published by the student affairs office. I understand what is expected of me with respect to properly citing sources, and how to avoid representing the work of others as my own. The material in this paper was written by me, except for such material that is quoted or indented and properly cited to indicate the sources of the material. I understand that using the words of others, and simply tagging the sentence, paragraph, or section with a tag to the copied source does not constitute proper citation and that if such material is used verbatim or paraphrased it must be specifically conveyed (such as through the use of quotation marks or indentation) together with the citation. I further understand that overuse of properly cited quotations to avoid conveying the information in my own words, while it will not subject me to disciplinary action, does convey to the instructor that I do not understand the material enough to explain it in my own words, and will likely result in a lesser grade on the paper.
		
		Signed: Hallgrimur David Egilsson
		
		\newpage

		\section{Workspace}

		For simplicity of this PoC (Proof of Concept) the system will only support 5 hard-coded patients and 2 hard-coded audit companies. The patients in the systems will be:

		Alice, Bob, Carol, David, Eve

		The first audit company is USC and should be able to audit the EHR records of Alice, Bob and Eve.

		The second audit company is UCLA and should be able to audit the EHR records of Carol and David.

		Python pajsonckage setup follow recommendations at: https://packaging.python.org/en/latest/tutorials/packaging-projects/

		EHR id generation:
		* Explain the chance of generating two ids that are the same.

		Database file:
		* current: linux/ MAC OS lockin with /tmp/ehr_db.json

		EHR REST API:
		* POST data should always be in application/json

		User IDs are just names for simplicity, we assume everyone has a uniqe name.

		System architecture visualization: https://structurizr.com/dsl

		Rest API specification: https://editor.swagger.io/

		RSA PKCS#1 OEAP max message length: https://datatracker.ietf.org/doc/html/rfc8017#section-7.1
			* Changed underlying hashing algorithm from SHA1 to SHA256 so max keysize is 190 bytes which is plenty enough (we just need 32 bytes for the 256 bit AES key)
		
		Using different keys for encryption & signing, probably more secure?

		PEM key format:
		*'PEM'*. (*Default*) Text encoding, done according to `RFC1421`_/`RFC1423`_.

		The reason for timestamp is to make sure adversary can't do replay attack constantly and see how ciphertext grows (they might learn a visit to the doctor happened).

		Instead of doing authentication inside the TLS channel, the TLS channel can be configured to use client keys as well, that would have been another option.

		Authenticating clients: TLS client cert vs application level authentication
		* TLS:
			* Built-in replay attack protection
			* Needs generation of certificates & ngnix config
		* Application level
			* Need to implement replay attack protection somehow (some timestamp shenanigans?)
			* More complex code
		
		Better programming to do the TLS stuff in Ngnix and connect to Flask through Gunicorn or similar WSGI.

		Name of doctor can be anything and is not verified, because it's not the focus of this project.

		doctor client examples:

		```
		python3 ehr.py CREATE alice mark -d 'Visit 8'
		python3 ehr.py DELETE alice john --ehr-id 73a6259260e4237019e1422c7b31adb0
		python3 ehr.py CHANGE alice mark --description 'Visit 8' --ehr-id dee291db65e4cf73ca02c152468cc674
		python3 ehr.py GET_RECORD alice john --ehr-id dee291db65e4cf73ca02c152468cc674
		python3 ehr.py GET_RECORDS bob john
		```
		
		\section{Introduction}

		\section{System architecture}

		\textit{Describe the system components (e.g., authentication server, audit server, etc.), their functionality, and communication patterns. Clearly describe how your system meets the five goals discussed above.}

		\section{Cryptographic components}

		\textit{discuss appropriate choice of specific cryptographic primitives to ensure the system supports the goals outlined above.}

		\textit{Describe the concrete encryption schemes and key management approaches to be used in your system}

		\section{Limitations of the system}

		\textit{Which challenges were not addressed?}
		
		\section{Example section}
		
		Example citation \cite{neuman2009challenges}
		
		\subsection{Example subsection 1}
		
		\subsection{Example subsection 2}
		
		\newpage
		
		\bibliography{citations.bib}
		
		\bibliographystyle{ieeetr}
		
		
	\end{flushleft}
	
\end{document}
